\documentclass[12point]{article}



\begin{document}

\section{Diffusion}
We model external virus as a concentration that diffuses and can be uptake or exported. We use the standard PhysiCell equations: 

\begin{eqnarray}
\frac{partial \rho }{\partial t} & = & 
D \nabla^2 \rho - \lambda \rho + 
\sum_{\textrm{cells }i} 
\delta( \vec{x} - \vec{x}_i ) \left[ 
V_i U_i \rho + E_i 
\right]
\end{eqnarray}
where $D$ is the diffusion coefficient, $\lambda$ is the decay rate, $U$ is the cell's uptake rate, and $E$ is the net export rate (see more below). 

We use a PDE like this for viral particles $V$, uncoated virus $U$ (which might be released by dead cells), viral RNA $R$ (might be released by dead cells), viral protein $P$ (might be released by dead cells), and also 
release assembled virus (used as a technical assist on viral internalization and export dynamics--all virions in this last equation are immediately transferred to the viral concentration. See below). 

\section{Internal virus model}
Let's define the following variables: 
\begin{enumerate}
\item 
$V$ are viruses that are bound to the cell surface and in the process of endocytosis / adsorption. 

\item 
$U$ are viruses that have been fully endocytoses and coated. 

\item 
$R$ is RNA that has been delivered from uncoated virus $U$

\item 
$P$ are (full sets) of viral proteins that have been synthesized from RNA $R$

\item 
$A$ are assembled virions that are ready for export. 

\end{enumerate}
Let's model just the internal processes, noting that PhysiCell automatically takes care of virus intake (source for $V$) and export (sink for $A$). 

Our first model of this will be a system of ODEs: 
\begin{eqnarray}
\frac{dV}{dt} & = & -r_U V \\
\frac{dU}{dt} & = & r_U V - r_P U \\
\frac{dR}{dt} & = & r_P R - \lambda_R R \\
\frac{dP}{dt} & = & r_S R - \lambda_P P - r_A P \\
\frac{dA}{dt} & = & r_A P
\end{eqnarray}

Next, we need to figure out net virus export. We use PhysiCell's standard form. If $\rho_A$ is the external concentration of exported (and assembled) viral particles, then: 
\begin{eqnarray}
\frac{d\rho_A}{dt} = ... \sum_{\textrm{cells } i} \delta( \vec{x}-\vec{x}_i ) E_{A,i}
\end{eqnarray}
where $\delta$ is the Dirac delta function centered at the cell's position
$\vec{x}_i$, and $E_{A,i}$ is the net export rate (in assembled virions per time). 

We use the following constitutive relation: 
\begin{eqnarray}
E_A & = & r_E A. 
\end{eqnarray}

As one final note: we want exported $A$ diffuse and be able to infect other cells. We do this with a ``cleanup'' step at each PhysiCell diffuison time step: in each voxel, we move all of $\rho_A$ to $\rho_V$ (the concentration 
of viral particles). 

\section{Cell apoptotic response (PD)}
We use a fairly standard Hill function: let $e$ be the effect, which we model as increasing with assembled viral particles: 
\begin{eqnarray}
e & - & \frac{ A^n  }{ A^n + A_{\frac{1}{2}}^n  }
\end{eqnarray}
where $A_{\frac{1}{2}}$ is the half max, and $n$ is the Hill coefficient. 

Then, the cell's apoptosis rate is 
\begin{eqnarray}
r_{\textrm{apoptosis}} = \overline{r}_\textrm{max} e. 
\end{eqnarray}

At death, some fraction $f$ of the cell's $V$, $U$, $R$, $P$, and $A$ are released into the microenvironment. These can potentially be individually specified. 

\section{Later}
Use a molecular-scale model of ACE2 to model how the uptake rate changes. 

Use a receptor trafficking model to also modulate virion uptake. 

Better parameter estimates of viral replication.

Add immune response with more cell types. Cell death can release immunostimulatory factors that bring in macrophages. Those cause more damage and feedback. Connect that to ARDS. 

\end{document}